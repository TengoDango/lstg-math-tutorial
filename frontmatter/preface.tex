\chapter{前言}

大概你群确实不需要一个专门的系统性的数学教程,
弹幕最常用的那套 ``三角函数+极坐标+参数方程+向量'' 体系
虽然高中才学到, 但是其核心理念是容易理解的, 网上资料也很丰富,
等真的需要用的时候群友教一下, 自己学一下也就会了, 群里氛围还挺好的.

不过还是写了这个教程, 倘若以后我不在群里了,
再有新人数学基础薄弱, 本教程或可提供一个相对简单的学习路线.
之前小摩老师写过数学教程, 虽然那个教程没写完,
而且阅读难度可能比较高, 但是我要向小摩老师表达敬意,
没有您就没有这个新的数学教程.

本教程将以 ``描述点的位置'' 为基点,
延伸到 ``描述点的运动'', ``描述曲线'' 等话题,
致力于涵盖制作弹幕需要的数学基础知识.
每一节可能会提出一些习题,
主要是对该节基础内容的考察 (较简单),
以及从实际弹幕衍生出的数学问题 (可能较难),
星号数量表示问题的大致难度, 没标星号表示没难度 (doge).
参考答案仅提供一种可行的思路, 读者有其他解法自然更好.

\textit{
  注意: 本教程会使用一些不规范的数学记号,
  在正式的数学交流场合不要误用哦(笑)
}

数学理论终究是空泛的, 与实际的弹幕制作仍有一段距离.
因此读者不必在数学理论上纠缠太多时间,
本教程也会尽量减少篇幅以节省读者时间.

若有读者想在本教程基础上修改完善, 可以在Github上
\href{https://github.com/TengoDango/lstg-math-tutorial}{获取源码}.
方便的话顺便告诉我一声, 我会很高兴的.

本人的能力和时间有限, 教程中难免有疏漏或错误,
欢迎读者批评指正. 如果想要就教程内容展开交流,
也欢迎联系我, 或者在LuaSTG交流群提出.
十分感谢您阅读本教程!
